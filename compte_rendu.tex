\documentclass{article}
\usepackage{graphicx} % Required for inserting images
\usepackage[table]{xcolor}
\usepackage{float}
\usepackage[a4paper, margin=2cm]{geometry}
\usepackage[T1]{fontenc}
\usepackage[french]{babel}
\usepackage{hyperref}
\usepackage{listings}
\usepackage{textcomp}
\usepackage{color}


\definecolor{codegreen}{rgb}{0,0.6,0}
\definecolor{codegray}{rgb}{0.5,0.5,0.5}
\definecolor{codepurple}{HTML}{C42043}
\definecolor{backcolour}{HTML}{F2F2F2}
\definecolor{bookColor}{cmyk}{0,0,0,0.90}  
\color{bookColor}

\lstset{upquote=true}

\lstdefinestyle{mystyle}{
    backgroundcolor=\color{backcolour},   
    commentstyle=\color{codegreen},
    keywordstyle=\color{codepurple},
    numberstyle=\numberstyle,
    stringstyle=\color{codepurple},
    basicstyle=\footnotesize\ttfamily,
    breakatwhitespace=false,
    breaklines=true,
    captionpos=b,
    keepspaces=true,
    numbers=left,
    numbersep=10pt,
    showspaces=false,
    showstringspaces=false,
    showtabs=false,
}
\lstset{style=mystyle}

\newcommand\numberstyle[1]{%
    \footnotesize
    \color{codegray}%
    \ttfamily
    \ifnum#1<10 0\fi#1 |%
}

\title{TP SGBD : \\Modèle de données}
\author{Marc Robison\\
	Nils Peteil\\
  Boucherie Matthieu}

\begin{document}

\maketitle

\subsection{Modèle conceptuel}

\begin{tt}
  Livre(\underline{id\_livre}, titre, résumés, dates\_de\_publication, nombre\_exemplaires,\\ date ajout, \#id\_Auteurs(Auteurs), \#id\_Editeur(Editeurs), \#id\_Cathégorie(Cathégories))

\smallskip
Auteurs(\underline{id\_auteurs}, Nom, Prénom, \#id\_livre(livre))

\smallskip
Editeur(\underline{id\_editeur}, nom\_editeur, \#id\_Livre(livre))

\smallskip
Cathégorie(\underline{id\_cathégorie}, nom\_cathégories, \#id\_Livre(livre))
\smallskip

Membre(\underline{id\_Membre}, Nom, Prénom, coordonnées, domicile, courrier, \#id\_Emprunts, \#id\_Avis)
\smallskip

Amendes(\underline{id\_Amendes}, prix, \#id\_Membres, \#id\_Livre)
\smallskip

Emprunts(\underline{id\_Emprunts}, date\_emprunt, date\_retour, \#id\_Membre, \#id\_Livre)
\smallskip

Avis(\underline{id\_Avis}, \#id\_Livre, Avis)
\smallskip

Reservation(\underline{id\_Reservation}, \#id\_Membre, \#id\_Livre)
\end{tt}


\bigskip
Le modèle conceptuel repose sur les éléments suivants :

\begin{enumerate}
  \item Livre : Représente les livres de la bibliothèque avec des informations comme le titre, le résumé, la date de publication et le nombre d'exemplaires. Il est lié aux \textbf{auteurs}, à un \textbf{éditeur} et à une \textbf{catégorie}.
  \item Auteurs : Chaque auteur a un identifiant, un nom et un prénom. Il est lié à un ou plusieurs \textbf{livres}.
\item Éditeur : Contient les informations sur les éditeurs, avec un lien vers les livres qu'ils ont publiés.
\item Catégorie : Permet de classer les livres en différentes catégories.
\item Membre : Représente les utilisateurs de la bibliothèque, avec leurs coordonnées et d'autres informations. Un membre peut être lié à des \textbf{emprunts}, des \textbf{avis}, des \textbf{réservations}, et des \textbf{amendes}.
\item Amendes : Gère les pénalités financières infligées aux membres pour les retards ou autres infractions. Chaque amende est liée à un \textbf{membre} et un \textbf{livre}.
\item Emprunts : Enregistre les prêts de livres aux membres, avec les dates d’emprunt et de retour.
\item Avis : Permet aux membres de laisser des avis sur les livres.
\item Réservation : Gère les réservations de livres par les membres.
\end{enumerate}





\end{document}
